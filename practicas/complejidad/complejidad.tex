%%%%%%%%%%%%%%%%%%%%%%%%%%%%%%%%%%%%%%%%%
% Programming/Coding Assignment
% LaTeX Template
%
% This template has been downloaded from:
% http://www.latextemplates.com
%
% Original author:
% Ted Pavlic (http://www.tedpavlic.com)
%
% Note:
% The \lipsum[#] commands throughout this template generate dummy text
% to fill the template out. These commands should all be removed when 
% writing assignment content.
%
% This template uses a Perl script as an example snippet of code, most other
% languages are also usable. Configure them in the "CODE INCLUSION 
% CONFIGURATION" section.
%
%%%%%%%%%%%%%%%%%%%%%%%%%%%%%%%%%%%%%%%%%

%----------------------------------------------------------------------------------------
%	PACKAGES AND OTHER DOCUMENT CONFIGURATIONS
%----------------------------------------------------------------------------------------

\documentclass[11pt, spanish]{article}

\usepackage[spanish]{babel}
\selectlanguage{spanish}
\usepackage[utf8]{inputenc}
\usepackage{fancyhdr} % Required for custom headers
\usepackage{lastpage} % Required to determine the last page for the footer
%\usepackage{extramarks} % Required for headers and footers
\usepackage[usenames,dvipsnames]{color} % Required for custom colors
\usepackage{graphicx} % Required to insert images
\usepackage{listings} % Required for insertion of code
\usepackage{courier} % Required for the courier font

\usepackage{blindtext}
\usepackage{amsmath}
\usepackage{amssymb}

\usepackage{cancel}

\usepackage[a4paper, total={6in, 8in}]{geometry}


%----------------------------------------------------------------------------------------
%	DOCUMENT STRUCTURE COMMANDS
%	Skip this unless you know what you're doing
%----------------------------------------------------------------------------------------

% Header and footer for when a page split occurs within a problem environment
\newcommand{\enterProblemHeader}[1]{
\nobreak\extramarks{#1}{#1 continued on next page\ldots}\nobreak
\nobreak\extramarks{#1 (continued)}{#1 continued on next page\ldots}\nobreak
}

% Header and footer for when a page split occurs between problem environments
\newcommand{\exitProblemHeader}[1]{
\nobreak\extramarks{#1 (continued)}{#1 continued on next page\ldots}\nobreak
\nobreak\extramarks{#1}{}\nobreak
}

\setcounter{secnumdepth}{0} % Removes default section numbers
\newcounter{homeworkProblemCounter} % Creates a counter to keep track of the number of problems

\newcommand{\homeworkProblemName}{}
\newenvironment{homeworkProblem}[1][Problem \arabic{homeworkProblemCounter}]{ % Makes a new environment called homeworkProblem which takes 1 argument (custom name) but the default is "Problem #"
\stepcounter{homeworkProblemCounter} % Increase counter for number of problems
\renewcommand{\homeworkProblemName}{#1} % Assign \homeworkProblemName the name of the problem
\section{\homeworkProblemName} % Make a section in the document with the custom problem count
\enterProblemHeader{\homeworkProblemName} % Header and footer within the environment
}{
\exitProblemHeader{\homeworkProblemName} % Header and footer after the environment
}

\newcommand{\problemAnswer}[1]{ % Defines the problem answer command with the content as the only argument
\noindent\framebox[\columnwidth][c]{\begin{minipage}{0.98\columnwidth}#1\end{minipage}} % Makes the box around the problem answer and puts the content inside
}

\newcommand{\homeworkSectionName}{}
\newenvironment{homeworkSection}[1]{ % New environment for sections within homework problems, takes 1 argument - the name of the section
\renewcommand{\homeworkSectionName}{#1} % Assign \homeworkSectionName to the name of the section from the environment argument
\subsection{\homeworkSectionName} % Make a subsection with the custom name of the subsection
\enterProblemHeader{\homeworkProblemName\ [\homeworkSectionName]} % Header and footer within the environment
}{
\enterProblemHeader{\homeworkProblemName} % Header and footer after the environment
}

%----------------------------------------------------------------------------------------

\begin{document}

\section*{Ejercicio 1}

\framebox{\begin{minipage}{0.98\columnwidth}
Probar utilizando las definiciones que $f \in  O(h)$, sabiendo que:

a) $f, h : \mathbb{N} \to \mathbb{N}$ son funciones tales que $f(n) = n^2 - 4n - 2$ y $h(n) = n^2$.

b) $f,g,h : \mathbb{N} \to \mathbb{N}$ son funciones tales que $g(n) = n^k, h(n) = n^{k+1} y f\in O(g)$.

c) $f,g,h : \mathbb{N} \to \mathbb{N}$ son funciones tales que $g(n) = log(n), h(n) = n y f \in O(g)$.
\end{minipage}}

\vspace{1em}
a) Quiero ver que $f \in O(h)$, por definición:
\[
f \in  O(h) \iff f \in \{f : \mathbb{N} \to \mathbb{N} | (\exists n_0, c)(\forall n > n_0) f(n) \leq ch(n)) \}
\] 

Basta con mostrar que existen $n_0$ y $c$ que satisfagan la condición. Tomo $c=1, n_0=1$.

Quiero ver que para todo $n$ mayor a $n_0$ vale que $f(n) \leq c*h(n)$. Para ello, voy a usar inducción. Verifico que la proposición vale para el caso $n_0=1$:
\[ 
f(1) = 1 - 4 - 2 = -5 \leq h(1) = 1 \iff -5 \leq 1
\]

Luego, basta con ver que $f(n) \leq ch(n) \Rightarrow f(n+1) \leq ch(n+1)$.
\begin{equation*}\begin{split}
f(n+1) &\leq ch(n+1)\\
\cancel{(n+1)^2} - 4(n+1) - 2 &\leq \cancel{(n+1)^2}\\
-4n - 6 &\leq 0
\end{split}\end{equation*}

Que vale para todo $n \in \mathbb{N}$ (al final no hacía falta hacer inducción, ¿no?).


\section*{Ejercicio 2}
\framebox{\begin{minipage}{0.98\columnwidth}
Determinar verdadero o falso.

\begin{itemize}
	\item a) $2^n = O(1)$
	\item b) $n = O(n!)$
	\item c) $n! = O(n^n)$
	\item d) $2^n = O(n!)$
	\item e) Para todo $i,j \in \mathbb{N}, i*n = O(j*n)$
	\item f) Para todo $k \in \mathbb{N}, 2^k = O(1)$
	\item g) $log(n) = O(n)$
	\item h) $n! = O(2^n)$
	\item i) $2^n*n^2 = O(3^n)$
	\item j) Para toda función $f: \mathbb{N} \to \mathbb{N}, f=O(f)$
\end{itemize}
\end{minipage}}

a) $2^n = O(1)$. Supongo que la afirmación vale. Luego, debería existir un $n_0$ y un $c$ tal que $2^n <= c$ para todo $n>n_0$. Tomo $n=max(c,n_0)$. Luego, es fácil ver que $2^{n} >= c$; lo cual es una contradicción que surge de suponer que $2^n = O(1)$. Luego, la afirmación es falsa.

\vspace{1em}
b) $n = O(n!)$.
La afirmación es verdadera. Tomo $c=1$, luego vale que 
\[
(\forall n \in \mathbb{N}) n <= n! \iff 1 <= n!/n \iff 1 <= (n-1)!
\]

\vspace{1em}
c) $n! = O(n^n)$.
La afirmación es verdadera. Vale que $(\forall n \in \mathbb{N})\ n! \leq n^n$. También vale que $n^n = O(n^n)$, luego $n! \leq n^n = O(n^n) \Rightarrow n! = O(n^n)$

\vspace{1em}
d) $2^n = O(n!)$

\vspace{1em}
e) Para todo $i,j \in \mathbb{N}, i*n = O(j*n)$

\vspace{1em}
f) Para todo $k \in \mathbb{N}, 2^k = O(1)$

\vspace{1em}
g) $log(n) = O(n)$

\vspace{1em}
h) $n! = O(2^n)$

\vspace{1em}
i) $2^n*n^2 = O(3^n)$

\vspace{1em}
j) Para toda función $f: \mathbb{N} \to \mathbb{N}, f=O(f)$

\end{document}