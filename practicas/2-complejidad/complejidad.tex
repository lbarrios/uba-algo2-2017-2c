%%%%%%%%%%%%%%%%%%%%%%%%%%%%%%%%%%%%%%%%%
% Programming/Coding Assignment
% LaTeX Template
%
% This template has been downloaded from:
% http://www.latextemplates.com
%
% Original author:
% Ted Pavlic (http://www.tedpavlic.com)
%
% Note:
% The \lipsum[#] commands throughout this template generate dummy text
% to fill the template out. These commands should all be removed when 
% writing assignment content.
%
% This template uses a Perl script as an example snippet of code, most other
% languages are also usable. Configure them in the "CODE INCLUSION 
% CONFIGURATION" section.
%
%%%%%%%%%%%%%%%%%%%%%%%%%%%%%%%%%%%%%%%%%

%----------------------------------------------------------------------------------------
%	PACKAGES AND OTHER DOCUMENT CONFIGURATIONS
%----------------------------------------------------------------------------------------

\documentclass[11pt, spanish]{article}

\usepackage[spanish]{babel}
\selectlanguage{spanish}
\usepackage[utf8]{inputenc}
\usepackage{fancyhdr} % Required for custom headers
\usepackage{lastpage} % Required to determine the last page for the footer
%\usepackage{extramarks} % Required for headers and footers
\usepackage[usenames,dvipsnames]{color} % Required for custom colors
\usepackage{graphicx} % Required to insert images
\usepackage{listings} % Required for insertion of code
\usepackage{courier} % Required for the courier font

\usepackage{blindtext}
\usepackage{amsmath}
\usepackage{amssymb}

\usepackage{cancel}

\usepackage{multicol}

\usepackage[a4paper, total={6in, 8in}]{geometry}


%----------------------------------------------------------------------------------------
%	DOCUMENT STRUCTURE COMMANDS
%	Skip this unless you know what you're doing
%----------------------------------------------------------------------------------------

% Header and footer for when a page split occurs within a problem environment
\newcommand{\enterProblemHeader}[1]{
\nobreak\extramarks{#1}{#1 continued on next page\ldots}\nobreak
\nobreak\extramarks{#1 (continued)}{#1 continued on next page\ldots}\nobreak
}

% Header and footer for when a page split occurs between problem environments
\newcommand{\exitProblemHeader}[1]{
\nobreak\extramarks{#1 (continued)}{#1 continued on next page\ldots}\nobreak
\nobreak\extramarks{#1}{}\nobreak
}

\setcounter{secnumdepth}{0} % Removes default section numbers
\newcounter{homeworkProblemCounter} % Creates a counter to keep track of the number of problems

\newcommand{\homeworkProblemName}{}
\newenvironment{homeworkProblem}[1][Problem \arabic{homeworkProblemCounter}]{ % Makes a new environment called homeworkProblem which takes 1 argument (custom name) but the default is "Problem #"
\stepcounter{homeworkProblemCounter} % Increase counter for number of problems
\renewcommand{\homeworkProblemName}{#1} % Assign \homeworkProblemName the name of the problem
\section{\homeworkProblemName} % Make a section in the document with the custom problem count
\enterProblemHeader{\homeworkProblemName} % Header and footer within the environment
}{
\exitProblemHeader{\homeworkProblemName} % Header and footer after the environment
}

\newcommand{\problemAnswer}[1]{ % Defines the problem answer command with the content as the only argument
\noindent\framebox[\columnwidth][c]{\begin{minipage}{0.98\columnwidth}#1\end{minipage}} % Makes the box around the problem answer and puts the content inside
}

\newcommand{\homeworkSectionName}{}
\newenvironment{homeworkSection}[1]{ % New environment for sections within homework problems, takes 1 argument - the name of the section
\renewcommand{\homeworkSectionName}{#1} % Assign \homeworkSectionName to the name of the section from the environment argument
\subsection{\homeworkSectionName} % Make a subsection with the custom name of the subsection
\enterProblemHeader{\homeworkProblemName\ [\homeworkSectionName]} % Header and footer within the environment
}{
\enterProblemHeader{\homeworkProblemName} % Header and footer after the environment
}

%----------------------------------------------------------------------------------------

\begin{document}

%----------------------------------------------------------------------------------------
% Ejercicio 1
%----------------------------------------------------------------------------------------
\section*{Ejercicio 1}

%
% Consigna
%
\framebox{\begin{minipage}{0.98\columnwidth}
Probar utilizando las definiciones que $f \in  O(h)$, sabiendo que:

a) $f, h : \mathbb{N} \to \mathbb{N}$ son funciones tales que $f(n) = n^2 - 4n - 2$ y $h(n) = n^2$.

b) $f,g,h : \mathbb{N} \to \mathbb{N}$ son funciones tales que $g(n) = n^k$, $h(n) = n^{k+1}$ y $f\in O(g)$.

c) $f,g,h : \mathbb{N} \to \mathbb{N}$ son funciones tales que $g(n) = log(n)$, $h(n) = n$ y $f \in O(g)$.
\end{minipage}}

%
% a)
%
\vspace{1em}
a) Quiero ver que $f \in O(h)$, por definición:
\[
f \in  O(h) \iff f \in \{f : \mathbb{N} \to \mathbb{N} | (\exists n_0, c)(\forall n > n_0) f(n) \leq ch(n)) \}
\] 

Basta con mostrar que existen $n_0$ y $c$ que satisfagan la condición. Tomo $c=1, n_0=1$.

Quiero ver que para todo $n$ mayor a $n_0$ vale que $f(n) \leq c*h(n)$. Para ello, voy a usar inducción. Verifico que la proposición vale para el caso $n_0=1$:
\[ 
f(1) = 1 - 4 - 2 = -5 \leq h(1) = 1 \iff -5 \leq 1
\]

Luego, basta con ver que $f(n) \leq ch(n) \Rightarrow f(n+1) \leq ch(n+1)$.
\begin{equation*}\begin{split}
f(n+1) &\leq ch(n+1)\\
\cancel{(n+1)^2} - 4(n+1) - 2 &\leq \cancel{(n+1)^2}\\
-4n - 6 &\leq 0
\end{split}\end{equation*}

Que vale para todo $n \in \mathbb{N}$.

%
% b)
%
\vspace{1em}
b) Quiero ver por definición que $f \in O(h)$, dado que $f \in O(g)$.

Se que vale $f \in  O(g)$:
\[
f \in  O(g) \iff f \in \{f : \mathbb{N} \to \mathbb{N} | (\exists n_0, c)(\forall n > n_0) f(n) \leq cg(n)) \}
\]

Y quiero mostrar que vale $f \in  O(h)$:
\[
f \in  O(h) \iff f \in \{f : \mathbb{N} \to \mathbb{N} | (\exists n_0, c)(\forall n > n_0) f(n) \leq ch(n)) \}
\]

Dado que vale $f \in  O(g)$, se que existe un par $n_0,c_0$ tal que para todo $n$ mayor a ese $n_0$ se cumple que $f(n) \leq c_0.g(n)$. Bastaría mostrar que $g \in O(h)$, lo que implicaría que existiría un par $n_1, c_1$ tal que para todo $n$ mayor a ese $n_1$ valdría que $g(n) \leq c_1.h(n)$, es decir que existirían $n_2=n_0+n_1$ y $c_2=c_1.c_0$, tal que para todo n mayor a $n_2$ valdría $f(n) \leq c_0.g(n) \leq c_2.h(n) \iff f(n) \leq c_2.h(n) \iff f \in O(h)$.

Quiero ver que $g \in O(h)$. Esto es lo mismo que decir que existe un par $n_0, c$ tal que para todo $n$ mayor a $n_0$ se cumple que $g(h) \leq c.h(n) \iff n^k \leq c.n^{k+1}$. Pero $n^k \leq c.n^k+1$ para todo n, para todo c. Luego, $g \in O(h)$. Finalmente, vale $f \in O(h)$.

%
% c)
%
\vspace{1em}
c) Quiero ver que $f \in O(h)$, dado que $g(n) = log(n)$, $h(n) = n$ y $f \in O(g)$.
Al igual que con el ejercicio anterior, siendo que $f \in O(g)$, bastaría con mostrar que $g \in O(h)$ para llegar a la conclusión de que $f \in O(h)$.

Quiero ver que $g \in O(h)$. Dado que $g(n) \leq h(n)$, que es lo mismo que $log(n) \leq n$, vale para todo n, al igual que sucedió con el ejercicio anterior, resulta que $g \in O(h)$. Luego, $f \in O(h)$.

%----------------------------------------------------------------------------------------
% Ejercicio 2
%----------------------------------------------------------------------------------------
\section*{Ejercicio 2}
\framebox{\begin{minipage}{0.98\columnwidth}
Determinar verdadero o falso.

\begin{itemize}
	\item a) $2^n = O(1)$
	\item b) $n = O(n!)$
	\item c) $n! = O(n^n)$
	\item d) $2^n = O(n!)$
	\item e) Para todo $i,j \in \mathbb{N}, i*n = O(j*n)$
	\item f) Para todo $k \in \mathbb{N}, 2^k = O(1)$
	\item g) $log(n) = O(n)$
	\item h) $n! = O(2^n)$
	\item i) $2^n*n^2 = O(3^n)$
	\item j) Para toda función $f: \mathbb{N} \to \mathbb{N}, f=O(f)$
\end{itemize}
\end{minipage}}

a) $2^n = O(1)$. Supongo que la afirmación vale. Luego, debería existir un $n_0$ y un $c$ tal que $2^n <= c$ para todo $n>n_0$. Tomo $n=max(c,n_0)$. Luego, es fácil ver que $2^{n} >= c$; lo cual es una contradicción que surge de suponer que $2^n = O(1)$. Luego, la afirmación es falsa.

\vspace{1em}
b) $n = O(n!)$.
La afirmación es verdadera. Tomo $c=1$, luego vale que 
\[
(\forall n \in \mathbb{N}) n <= n! \iff 1 <= n!/n \iff 1 <= (n-1)!
\]

\vspace{1em}
c) $n! = O(n^n)$.
La afirmación es verdadera. Vale que $(\forall n \in \mathbb{N})\ n! \leq n^n$. También vale que $n^n = O(n^n)$, luego $n! \leq n^n = O(n^n) \Rightarrow n! = O(n^n)$

\vspace{1em}
d) $2^n = O(n!)$. La afirmación es verdadera. $2^n$ es $(2 * 2 * \cdots * 2)$, es decir, $\prod_{i=1}^{n}2$. Por otro lado $O(n!)$ es $\prod_{i=1}^{n}i$. Quiero ver que a partir de un determinado $n_0$, y para una determinada $c$, vale que \[
2^n \leq c*n! \iff \prod_{i=1}^{n}2 \leq c*\prod_{i=1}^{n}i
\]
Tomo $c=2$.
\begin{equation*}\begin{split}
2^n &\leq 2*n!\\
\prod_{i=1}^{n}2 &\leq 2*\prod_{i=1}^{n}i\\
1 &\leq \frac{2*\prod_{i=1}^{n}i}{\prod_{i=1}^{n}2}\\
1 &\leq \frac{\cancel{2}*\cancel{1}*\prod_{i=2}^{n}i}{\cancel{2}* \prod_{i=2}^{n}2}\\
1 &\leq \prod_{i=2}^{n}{\frac{i}{2}}\\
\end{split}\end{equation*}

Dado que para todo $i\geq2$ vale que $\frac{i}{2}\geq1$, el producto de esos términos es mayor o igual a 1, por lo que la proposición vale; de lo cual se desprende que $2^n \leq 2*n!$.

\vspace{1em}
e) Para todo $i,j \in \mathbb{N}, i*n = O(j*n)$. La afirmación es verdadera. Quiero ver que, \emph{dados} $i_0,j_0 \in \mathbb{N}$ \ existen $c, n_0$ tales que para todo $n>n_0$ vale $i_0*n \leq c*j_0*n$. Tomo $c=i_0$. Luego, cualquiera sea $j_0$, vale para todo n que $i_0*n \leq i_0*j_0*n \iff 1 \leq j_0$. Finalmente, la afirmación vale para cualquier par $i,j \in \mathbb{N}$,

\vspace{1em}
f) Para todo $k \in \mathbb{N}, 2^k = O(1)$. La afirmación es verdadera. Sea $k_0 \in \mathbb{N}$, quiero ver existen $n_0, c \in \mathbb{N}$ tales que $2^{k_0} \leq c*1$ para todo $n\geq n_0$. Tomo $c=2^{k_0}$. Luego, vale para todo n que $2^{k_0} \leq 2^{k_0}$. Finalmente, la afirmación vale para todo k.

\vspace{1em}
g) $log_2(n) = O(n)$. La afirmación es verdadera. Quiero ver que existen $n_0, c$ tales que para $n\geq n_0$ vale $log_2(n) \leq c*n$.
\[
log_2(n) \leq c*n \iff 2^{log_2(n)} \leq 2^{c*n} \iff n \leq 2^{c*n}
\]

Quiero ver que vale $P(n) = n\leq 2^{c*n}$, verifico que vale para el caso base $n_0=1$,
\[
P(1): 1\leq 2^{c*1} \iff 1\leq 2^c
\]

Quiero ver que si vale para $n$, entonces vale para $n+1$.
\[
P(n+1): (n+1) \leq 2^{c*(n+1)}
\iff n+1 \leq 2^{c*n+c}
\iff n+1 \leq 2^{c*n} * 2^{c}
\]
\[
\iff \{
n \leq 2^{c*n} 
\land
\iff 1 \leq 2^{c}
\}
\]

$n \leq 2^{c*n}$ vale por hipótesis inductiva, y $1 \leq 2^{c}$ vale para todo $c$. Luego, $P(n)$ vale para todo $n>1$. Finalmente, vale $log_2(n) \leq c*n$ para todo $n,c \in \mathbb{N}$.

\vspace{1em}
h) $n! = O(2^n)$. La afirmación es falsa. Quiero ver que dados cualesquiera $n_0, c \in \mathbb{N}$, no es posible satisfacer que para todo $n\geq n_0$ valga $n! \leq c*2^n \iff \prod_{i=1}^{n}i \leq c*\prod_{i=1}^{n}2$. Para que esta afirmación sea verdadera, debería valer que \[
\frac{\prod_{i=1}^{n}i}{\prod_{i=1}^{n}2} \leq c
\iff
\prod_{i=1}^{n}{\frac{i}{2}} \leq c
\]

Pero $\frac{i}{2} \geq 2$ para todo $i\geq4$, luego $\prod_{i=1}^{n}{\frac{i}{2}}$ es una productoria en donde todos los términos $i>4$ son mayores a 1; es decir que el límite de esta productoria cuando n tiende a infinito, es infinito. Lo significa que dada una constante cualquiera c, siempre va a existir un n tal que la productoria sea mayor a la constante, por lo que no existe ningún par $n_0, c \in \mathbb{N}$ que haga que la afirmación valga.

\vspace{1em}
i) $2^n*n^2 = O(3^n)$. La afirmación es verdadera.

Quiero ver que existe un par $c_0, n_0$ tal que para todo $n>n_0$ valga
\[
2^n*n^2 \leq c_0*3^n
\]
Esto es lo mismo que decir que 
\[
n^2 \leq c_0*\frac{3^n}{2^n}
\]
\[
n^2 \leq c_0*(\frac{3}{2})^n
\]

Supongamos que $c_0=1$, vamos a probar que vale $P(n): n^2 \leq (\frac{3}{2})^n$ por inducción.

\[
P(1): 1^2 \leq (\frac{3}{2})^1 \iff 1 \leq \frac{3}{2}
\]

\[ P(n) \rightarrow P(n+1): \]
\[ (n+1)^2 \leq (\frac{3}{2})^{(n+1)} \]
\[ n^2 + 2n + 1 \leq (\frac{3}{2})^n*\frac{3}{2} \]
\[ n^2 + 2n + 1 \leq (\frac{3}{2})^n*(\frac{2}{2}+\frac{1}{2}) \]
\[ n^2 + 2n + 1 \leq (\frac{3}{2})^n + \frac{1}{2}*(\frac{3}{2})^n \]

Por HI vale que $n^2 \leq (\frac{3}{2})^n$. Restaría ver que $2n + 1 \leq \frac{1}{2}*(\frac{3}{2})^n$.
Para ello, vamos a volver a aplicar inducción. Queremos probar que vale $Q(n): 4n + 2 \leq (\frac{3}{2})^n$ a partir de algún $n>n_0$. Tomamos $n_0=50$.

\[
Q(50): 4*50 + 2 \leq (\frac{3}{2})^{50}
\]

\[
202 \leq 637621500
\]

Por lo que el caso base vale. Queremos ver si vale el paso inductivo: 

\[
Q(n) \rightarrow Q(n+1): 4*(n+1) + 2 \leq \frac{3}{2}^{n+1}
\]
\[
4n + 4 + 2 \leq (\frac{3}{2})^{n} * (\frac{3}{2})
\]
\[
4n + 4 + 2 \leq (\frac{3}{2})^{n} + \frac{1}{2}*(\frac{3}{2})^n
\]

Pero por HI sabemos que vale $4n + 2 \leq (\frac{3}{2})^{n}$, por lo que bastaría ver que vale $4 \leq \frac{1}{2}*(\frac{3}{2})^n$. Esto es sencillo, ya que 4 es una constante, y $(\frac{3}{2})^n$ es una función creciente, por lo que si vale para un determinado $n_0$, entonces vale para cualquier $n>n_0$. En particular, vale para $n_0=50$.
\[
4 \leq \frac{1}{2}*(\frac{3}{2})^{50}
\]
\[
4 \leq 318810750
\]

Finalmente, dado que vale Q(n) para $n_0=50$, entonces vale P(n) para $n_0=50$. Luego, existen $c_0=1, n_0=50$ a partir de los cuales vale la proposición, por lo que vale que $2^n*n^2 = O(3^n)$.


\vspace{1em}
j) Para toda función $f: \mathbb{N} \to \mathbb{N}, f=O(f)$\\

Quiero ver que existe un par $c_0,n_0$ tal que $f(n) \leq c_0*f(n)$.\\

Elijo $c_0=2$. Luego, quiero ver que:
\[ f(n) \leq 2f(n) \iff 0 \leq 2f(n)-f(n) \iff 0 \leq f(n) \]

Y dado que f va de naturales en naturales, $0 \leq f(n)$ vale para cualquier n.

%----------------------------------------------------------------------------------------
% Ejercicio 2
%----------------------------------------------------------------------------------------
\section*{Ejercicio 3}
\framebox{\begin{minipage}{0.98\columnwidth}
	a) ¿Qué significa, intuitivamente, $O(f) \subseteq O(g)$? ¿Qué se puede concluir acerca del crecimiento de $f$ y $g$ cuando, simultáneamente, tenemos $O(f) \subseteq O(g)$ y $O(g) \subseteq O(f)$?.\\
	
	b) ¿Cómo ordena por inclusión las siguientes familias de funciones?
	\begin{multicols}{3}
	\begin{itemize}
		\item $O(1)$
		\item $O(x+1)$
		\item $O(x^2)$
		\item $O(\sqrt{x})$
		\item $O(1/x)$
		
		\item $O(x^x)$
		\item $O(\sqrt{2})$
		\item $O(log~x)$
		\item $O(log^2 x)$
		\item $O(log(x^2))$
		
		\item $O(log~log~x)$
		\item $O(x!)$
		\item $O(log(x!))$
		\item $O(x~log~x)$
		\item $O(1 + sin^2~x)$
	\end{itemize}
	\end{multicols}
	\vspace{1em}
\end{minipage}}
\vspace{1em}

% TODO
% TODO: A)
% TODO
a) Intuitivamente, $O(f) \subseteq O(g)$ significa que para todas las $f$ tales que $f \in O(f)$, vale que $f \in O(g)$. Si tenemos que $O(f) \subseteq O(g)$ y $O(g) \subseteq O(f)$, esto es como decir que $f \in O(f) \rightarrow f \in O(g)$ y que $g \in O(g) \rightarrow g \in O(f)$, es decir que $O(g) = O(f)$.\\

b) \[ O(1)
\subseteq O(\sqrt{2})
\subseteq O(1+sin^2~x)
\subseteq O(1/x)
\subseteq O(log~log~x)
\subseteq O(log~x)
\]
\[
\cdots
\subseteq O(log~x^2)
\subseteq O(log^2~x)
\subseteq O(\sqrt{x})
\subseteq O(x+1)
\subseteq O(log(x!))
\]
\[
\cdots
\subseteq O(x~log~x)
\subseteq O(x^2)
\subseteq O(x!)
\subseteq O(x^x)
\]

\end{document}